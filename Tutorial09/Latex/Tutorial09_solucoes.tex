\documentclass[a4paper,twoside,12pt]{article}
\usepackage[top=1.5cm,left=1.5cm,right=1.5cm,bottom=2cm]{geometry}

\usepackage[utf8]{inputenc}
\usepackage[T1]{fontenc}
\usepackage{lmodern}
\usepackage[brazilian]{babel}

\usepackage{graphicx}
\usepackage{subfig}
\usepackage{psfrag}
\usepackage{cite}
\usepackage[cmex10]{amsmath}
\usepackage{amssymb}
\usepackage{upgreek}
\usepackage{microtype}
\usepackage{titling}
\usepackage{courier}
\usepackage{url}
\usepackage{hyperref}
\hypersetup{
	colorlinks=true,
	linkcolor=blue,
	urlcolor=blue
}

% Use the 'listings' package to add code and the 'color' package to generate the highlights
\usepackage{listings}
\usepackage{color}
\definecolor{grey}{rgb}{0.6,0.6,0.6}
\definecolor{codegreen}{rgb}{0,0.6,0}
\definecolor{codegray}{rgb}{0.5,0.5,0.5}
\definecolor{codepurple}{rgb}{0.58,0,0.82}
\definecolor{backcolour}{rgb}{0.95,0.95,0.92}
\lstset{language=Python,				% set it to Python language highlighting
	backgroundcolor=\color{backcolour},
	commentstyle=\color{codegreen},
	keywordstyle=\color{magenta},
	numberstyle=\tiny\color{codegray},
	stringstyle=\color{codepurple},
	basicstyle=\ttfamily\small,		% set size of fonts
	breaklines=true,				% set automatic line breaking
	linewidth=\textwidth,			% set size of code box
	showstringspaces=false}			% show spaces as underscores only inside strings


\title{\vspace{-2cm} Processamento Digital de Sinais\\ Soluções para Tutorial 09 - Convolução}
\author{\href{https://github.com/fchirono/AulasDSP}{Fabio Casagrande Hirono}}

\begin{document}
	\date{}
	\maketitle
	

%\setcounter{section}{1}
\section{Convolução}\label{sec:Convolution}
\setcounter{subsection}{1}

\subsection{Tarefa 1}
A solução é $Q = L + M - 1 = 103$.

\subsection{Tarefa 2}
- 

\subsection{Tarefa 3}
- 

\subsection{Tarefa 4}
Os três resultados devem mostrar a mesma saída, pelo menos até a precisão numérica do sistema.
\begin{figure}[h!]
	\centering
	\includegraphics[width = 0.6\textwidth]{VariantesConv.png}
	\caption{Resultados da convolução para a implementação no domínio do tempo, no domínio DFT e a função {\tt numpy.convolve()}.}
\end{figure}

\section{Aplicações de Convolução}
\subsection{Tarefa 1}

\begin{figure}[h]
	\centering
	\includegraphics[width = 0.6\textwidth]{RespImpulso.png}
	\caption{Gráfico da resposta ao impulso.}
\end{figure}

Você pode ver as primeiras reflexões discretas causadas por superfícies refletoras (paredes), seguidas pela curva de de.

\subsection{Tarefas 2-3}
É fortemente aconselhável usar a função {\tt convdft}, pois é muito mais rápida que a função {\tt convtime}. Você pode economizar mais algum tempo usando apenas alguns segundos do sinal de fala. As saídas dos métodos de convolução devem ser todas iguais. O efeito de reverberação que foi adicionado pela convolução do sinal com a resposta ao impulso dada deve ser claramente audível. Lembre-se de que os arrays Numpy devem ter sua amplitude normalizada antes de serem convertidos para arquivos {\tt.wav} para evitar distorção!

\section{Convolução Overlap-Add}

\subsection{Tarefa 1}
-

\subsection{Tarefa 2}
O resultado da operação deve produzir uma matriz $10 \times 10$ {\tt X} com os valores do vetor {\tt x}. Isso pode ser alcançado simplesmente lendo fragmentos de comprimento {\tt N = 10} de {\tt x} e escrevendo-os nas linhas de {\tt X}.

\subsection{Tarefa 3}
Certifique-se de inicializar a matriz {\tt Y} com as dimensões corretas. As linhas de {\tt Y} são mais longas do que as linhas de {\tt X} para acomodar os fragmentos estendidos obtidos da convolução (veja Seção \ref{sec:Convolution}).

\subsection{ Tarefa 4}
Aqui é importante entender que você tem que \textbf{somar} cada fragmento de {\tt Y} ao segmento correspondente do vetor {\tt y}, para garantir que contribuções ao final do segmento anterior tenham um efeito no segmento atual. (É por isso que o método se chama ``Overlap-Add''!)

\begin{figure}[h!]
	\centering
	\includegraphics[width = 0.6\textwidth]{OverlapAdd.png}
	\caption{Resultados da convolução para a implementação Overlap-Add.}
\end{figure}

\subsection{Tarefa 5}
%Aqui é importante inicializar {\tt Y} com zeros, de modo que quando o segundo quadro for escrito, você apenas altere as primeiras 48 amostras da segunda linha em {\tt Y}.
Novamente, certifique-se de inicializar a matriz {\tt Y} com as dimensões corretas para o problema.

\end{document}
