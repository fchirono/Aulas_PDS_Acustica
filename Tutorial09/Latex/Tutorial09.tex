\documentclass[a4paper,twoside,12pt]{article}
\usepackage[top=1.5cm,left=1.5cm,right=1.5cm,bottom=2cm]{geometry}

\usepackage[utf8]{inputenc}
\usepackage[T1]{fontenc}
\usepackage{lmodern}
\usepackage[brazilian]{babel}

\usepackage{graphicx}
\usepackage{subfig}
\usepackage{psfrag}
\usepackage{cite}
\usepackage[cmex10]{amsmath}
\usepackage{amssymb}
\usepackage{upgreek}
\usepackage{microtype}
\usepackage{titling}
\usepackage{courier}
\usepackage{url}
\usepackage{hyperref}
\hypersetup{
	colorlinks=true,
	linkcolor=blue,
	urlcolor=blue
}

% Use the 'listings' package to add code and the 'color' package to generate the highlights
\usepackage{listings}
\usepackage{color}
\definecolor{grey}{rgb}{0.6,0.6,0.6}
\definecolor{codegreen}{rgb}{0,0.6,0}
\definecolor{codegray}{rgb}{0.5,0.5,0.5}
\definecolor{codepurple}{rgb}{0.58,0,0.82}
\definecolor{backcolour}{rgb}{0.95,0.95,0.92}
\lstset{language=Python,				% set it to Python language highlighting
	backgroundcolor=\color{backcolour},
	commentstyle=\color{codegreen},
	keywordstyle=\color{magenta},
	numberstyle=\tiny\color{codegray},
	stringstyle=\color{codepurple},
	basicstyle=\ttfamily\small,		% set size of fonts
	breaklines=true,				% set automatic line breaking
	linewidth=\textwidth,			% set size of code box
	showstringspaces=false}			% show spaces as underscores only inside strings


\title{\vspace{-2cm} Processamento Digital de Sinais\\ Tutorial 09 - Convolução}
\author{\href{https://github.com/fchirono/AulasDSP}{Fabio Casagrande Hirono}}

\begin{document}
	\date{}
	\maketitle
	
	\section*{Objetivos do Tutorial}
	Ao final desta sessão, você será capaz de:
	\begin{itemize}
		\item Calcular a convolução entre duas sequências,
		\item Utilizar a convolução para adicionar reverberação artificial a um sinal de áudio.
	\end{itemize}
	
	\begingroup
	\let\clearpage\relax
	\tableofcontents
	\endgroup
	
	\section{Convolução}
\subsection{Revisão}\label{sec:SignalDefs}
Dado um sinal de tempo discreto $x[n]$ de comprimento $L$ amostras, e um segundo sinal de tempo discreto $h[n]$ de comprimento $M$, a convolução $y[n]$ de $x[n]$ e $h[n]$ é definida como
\begin{equation}\label{eq:convolution}
	y[n] = x[n] \ast h[n] = \sum_{m=0}^{M-1}h[m]\cdot x[n-m].
\end{equation}

\subsection{Tarefas}
\begin{enumerate}
	\item Questão teórica: sejam {\tt x} e {\tt h} vetores de comprimento {\tt L=100} e {\tt M=4}, respectivamente. Qual será o comprimento do vetor {\tt y} resultante da convolução de {\tt x} e {\tt h}? Forneça uma solução geral para o comprimento {\tt Q} de {\tt y}.
	\item Escreva uma função Python {\tt y = convtempo(x, h)} que implemente a equação (\ref{eq:convolution}) (não use a DFT!).
	\item Escreva uma segunda função {\tt y = convdft(x, h)} para realizar a convolução com DFT usando a função DFT que você criou no Tutorial 3. Lembre-se de que o número de pontos da DFT deve ser igual ao comprimento da saída. Se houver problemas com sua função DFT, use a funçã {\tt numpy.fft.fft()}.
	\item Teste as funções criadas nas Tarefas 2 e 3 usando {\tt x = [3, 4, 5]} e {\tt h = [1, 2, 3, 4]} como entradas. O pacote Numpy fornece a função {\tt numpy.convolve()} para realizar uma convolução. Compare as saídas das três funções. Elas são idênticas?
\end{enumerate}

	
	\section{Aplicações de Convolução}
A convolução pode ser usada para implementar reverberação artificial de um sinal gravado em condições \emph{anecóicas} usando uma resposta ao impulso medida em um espaço acústico.

\subsection{Tarefas}
\begin{enumerate}
	\item Leia o arquivo de áudio {\tt C4DM\_GreatHall\_Omni\_x00y05.wav} \footnote{Obtido em \url{http://isophonics.net/content/room-impulse-response-data-set}}, que contém uma resposta ao impulso medida em um salão, e escolha um sinal de entrada: o arquivo {\tt voz.wav} ou o arquivo {\tt cuica.wav}. A frequência de amostragem é de 48 kHz nos três casos. Plote a resposta ao impulso e pense sobre o que pode ter causado as principais características no gráfico entre 0s e 0,25 s.
	\item Execute a convolução dos arquivos de áudio escolhidos usando sua função {\tt convdft()} e {\tt numpy.convolve()}. \textbf{Aviso}: Dependendo do desempenho do computador no qual o script é executado, este processo pode levar até vários minutos.
	\item Auralize o sinal de saída e ouça o efeito resultante.
\end{enumerate}


	\section{OPCIONAL: Convolução Overlap-Add}
A convolução tipo sobreposição-e-soma (\emph{Overlap-Add}) é uma implementação da convolução adequada para aplicações em tempo real porque divide o sinal de entrada em quadros relativamente pequenos e, em seguida, executa uma convolução para cada quadro para adicioná-los posteriormente.

\subsection{Revisão}
Para aplicar o método Overlap-Add, a sequência $x[n]$ definida na Seção \ref{sec:SignalDefs} precisa ser subdividida em quadros com o comprimento escolhido $N$:
\begin{equation*}
	x_k[n] =
	\begin{cases}
		x[n+k\cdot N], &  n=0\ldots N-1 \\
		0, & \textrm{caso contrário},
	\end{cases}
\end{equation*}
onde $k$ denota o índice do quadro. O número total $K$ de quadros é $K=\lceil \frac{L}{N}\rceil$, onde $\lceil\cdot\rceil$ denota o primeiro inteiro maior que o argumento.

\par Isso então leva a
\begin{equation*}
	x[n] = \sum_{k=0}^{K-1}x_k[n-k\cdot N].
\end{equation*}

Esta subdivisão de $x[n]$ permite realizar a convolução para cada quadro individualmente para depois recombiná-los, em vez de executar uma única convolução longa. O resultado da convolução é então dado por
\begin{equation*}
	y[n] = \left(\sum_{k=0}^{K-1}x_k[n-k\cdot N]\right)\ast h[n] = \sum_{k=0}^{K-1}\sum_{m=0}^{M-1}h[m]\cdot x_k[n-k\cdot N-m]=\sum_{k=0}^{K-1}y_k[n-k\cdot N].
\end{equation*}

\subsection{Tarefas}
\begin{enumerate}
	\item Gere um vetor {\tt x} de comprimento {\tt L = 100}, contendo apenas o valor $1$ em todos os elementos. Em seguida, gere um vetor {\tt h=[1, 1, 1, 1]}, representando a resposta ao impulso de um sistema.
	\item Divida {\tt x} em quadros de comprimento {\tt N=10} e salve-os em uma matriz {\tt X}.
	\item Faça a convolução de cada vetor linha de {\tt X} com a resposta ao impulso {\tt h}, usando sua função {\tt convdft}, e salve os resultados em uma segunda matriz {\tt Y}. Certifique-se de que {\tt Y} tem as dimensões apropriadas.
	\item Finalmente, componha o sinal {\tt y} a partir da matriz {\tt Y}, usando o método Overlap-Add. Você pode verificar seus resultados de sua convolução Overlap-Add comparando-a com uma convolução de bloco único usando {\tt numpy.convolve()}.
	\item Refaça as tarefas 2-4 usando {\tt N = 52}. Observe as implicações disso, já que o último quadro do sinal será mais curto que {\tt N}.
	\item Escreva uma função {\tt y = OverlapAdd(x, h, N\_DFT)} executando a convolução de {\tt x} e {\tt h} com o método Overlap-Add. Certifique-se de que ela pode lidar com sinais que não podem ser divididos exatamente em quadros de comprimento {\tt N}. Note que {\tt N} é definido de forma única como $N = N_\textrm{DFT} - M + 1$.
\end{enumerate}
\end{document}
