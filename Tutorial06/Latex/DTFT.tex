\section{Transformada de Fourier de Tempo Discreto (DTFT)}

\subsection{Revisão}

A Transformada de Fourier de Tempo Discreto (DTFT) é definida da seguinte forma:

\begin{equation}\label{eq:DTFT}
	X(\Omega) = \sum_{n=-\infty}^{\infty}x[n]\cdot e^{-jn\Omega}, \qquad \Omega = 2\pi\frac{f}{f_s}\in[-\pi,\pi].
\end{equation}

Note que $X(\Omega)$ é uma função periódica, com período $2\pi$.

\subsection{DTFT de Sinais Não-Periódicos}

Leia os arquivos {\tt A1.wav} e {\tt A2.wav} usando o módulo {\tt scipy.io.wavfile}. Para incluir este módulo, use o comando:

\begin{lstlisting}[frame=single]
from scipy.io import wavfile
\end{lstlisting}

A função {\tt wavfile.read} pode ser usada para ler um arquivo .WAV de 16 bits por amostra no ambiente de trabalho da seguinte forma:

\begin{lstlisting}[frame=single]
# le a frequencia de amostragem e valores das amostras do arquivo .WAV
fs_wav, sinal_int16 = wavfile.read("Exemplo.wav")

# confirma o tipo de dados obtidos do arquivo wav
sinal_int16.dtype

# normalizacao dos dados de audio de int16 para ponto flutuante
N_bits = 16 
sinal_float = sinal_int16/(2.**(N_bits - 1)-1)
\end{lstlisting}

\paragraph{Tarefas}

\begin{enumerate}
	
	\item Mostre analiticamente que a DTFT é periódica, com período $2\pi$. \textbf{DICA}: Calcule $X(\Omega+2\pi)$ usando a equação (\ref{eq:DTFT}).
	
	\item Ouça ambos os sinais {\tt A1} e {\tt A2} usando seu reprodutor de mídia local. Eles têm a mesma frequência fundamental?
	
	\item Escreva uma função {\tt DTFT(x, Omega)} que implemente a equação (\ref{eq:DTFT}). O vetor de entrada {\tt Omega} da função contém as frequências angulares normalizadas em radianos/amostra a ser usadas na análise. Note que, embora a DTFT teoricamente tenha uma resolução de frequência infinita, em uma implementação computacional só podemos especificar uma quantidade limitada de frequências discretas.
	
	\item Crie um vetor {\tt Omega} contendo 4096 pontos igualmente espaçados no intervalo $[-\pi, \pi]$ definindo um conjunto de frequências angulares normalizadas em radianos/amostra, e use a função recém-implementada para calcular os espectros de Fourier de ambos os arquivos {\tt .wav}. Plote o valor absoluto de ambos os resultados em cores diferentes. 
	
	\item Qual é a resolução de frequência resultante em Hz da análise de Fourier realizada no item anterior?
	
	\item Que aspectos visíveis nos dois espectros de amplitude podem confirmar a sua observação na Tarefa 2?
	
	\item Defina um novo vetor {\tt f} (baseado em {\tt Omega}) que contenha as frequências correspondentes em Hertz. Plote o espectro de amplitude de {\tt A1.wav} e use {\tt f} para o eixo ``x''.
	
	\item Agora redefina {\tt Omega} para 4096 frequências angulares normalizadas uniformemente espaçadas no intervalo $[-3\pi,3\pi]$, calcule e plote a (magnitude da) DTFT para esta nova faixa de frequências normalizadas. Quais efeitos podem ser observados nos novos espectros de amplitude, e por que isso ocorre?
\end{enumerate}