\documentclass[a4paper,twoside,12pt]{article}
\usepackage[top=1.5cm,left=1.5cm,right=1.5cm,bottom=2cm]{geometry}

\usepackage[utf8]{inputenc}
\usepackage[T1]{fontenc}
\usepackage{lmodern}
\usepackage[brazilian]{babel}

\usepackage{graphicx}
\usepackage{subfig}
\usepackage{psfrag}
\usepackage{cite}
\usepackage[cmex10]{amsmath}
\usepackage{amssymb}
\usepackage{upgreek}
\usepackage{microtype}
\usepackage{titling}
\usepackage{courier}
\usepackage{url}
\usepackage{hyperref}
\hypersetup{
	colorlinks=true,
	linkcolor=blue,
	urlcolor=blue
}
\usepackage[makeroom]{cancel}

% Use the 'listings' package to add code and the 'color' package to generate the highlights
\usepackage{listings}
\usepackage{color}
\definecolor{grey}{rgb}{0.6,0.6,0.6}
\definecolor{codegreen}{rgb}{0,0.6,0}
\definecolor{codegray}{rgb}{0.5,0.5,0.5}
\definecolor{codepurple}{rgb}{0.58,0,0.82}
\definecolor{backcolour}{rgb}{0.95,0.95,0.92}
\lstset{language=Python,				% set it to Python language highlighting
	backgroundcolor=\color{backcolour},
	commentstyle=\color{codegreen},
	keywordstyle=\color{magenta},
	numberstyle=\tiny\color{codegray},
	stringstyle=\color{codepurple},
	basicstyle=\ttfamily\small,		% set size of fonts
	breaklines=true,				% set automatic line breaking
	linewidth=\textwidth,			% set size of code box
	showstringspaces=false}			% show spaces as underscores only inside strings

\title{\vspace{-2cm} Processamento Digital de Sinais e Aplicações em Acústica\\ Soluções para Tutorial 6 - Transformada de Fourier de Tempo Discreto (DTFT) e Transformada Discreta de Fourier (DFT)}
\author{\url{https://github.com/fchirono/Aulas_PDS_Acustica}}

\begin{document}
	\date{}
	\maketitle

%\setcounter{section}{1}
\section{DTFT}
\setcounter{subsection}{1}

\subsection*{Tarefa 1}
Mostre que a DTFT é periódica:
\begin{align*}
	X(\Omega+2\pi) & =\sum_{n=-\infty}^{n=\infty}x[n] \cdot e^{-jn(\Omega+2\pi)} \\
	& = \sum_{n=-\infty}^{n=\infty}x[n] \cdot e^{-jn\Omega} \cdot e^{-jn2\pi} \\
%	& e^{-jn2\pi} = (e^{-j2\pi})^n = 1^n = 1 \\ \\
	& = \sum_{n=-\infty}^{n=\infty}x[n] \cdot e^{-jn\Omega} \cdot \cancelto{1}{e^{-jn2\pi}} \\
	& = \sum_{n=-\infty}^{n=\infty}x[n] \cdot e^{-jn\Omega} = X(\Omega)
\end{align*}

\subsection*{Tarefa 2}
Não, o segundo arquivo possui uma freqência fundamental mais alta.

\subsection*{Tarefa 3}
-

\subsection*{Tarefa 4}
Ver Figura \ref{fig:DTFTA1A2}.

\begin{figure}[h!]
	\centering
	\includegraphics[width = \textwidth]{DTFTA1A2.png}
	\caption{DTFT dos sinais {\tt A1.wav} e {\tt A2.wav}.}
	\label{fig:DTFTA1A2}
\end{figure}

\subsection*{Tarefa 5}
$\Delta f = \frac{f_S}{4096 - 1} = 5.38$ Hz $=0.00153$ rad/amostra.

\subsection*{Tarefa 6}
Sim, o espectro harmônico de {\tt A2.wav} está mais espalhado, comparado ao espectro de {\tt A1.wav}. Uma outra pista é que a frequência fundamental é mais alta.

\subsection*{Tarefa 7}

Ver Figura \ref{fig:DTFTA1Hz}.

\begin{figure}[h!]
	\centering
	\includegraphics[width = \textwidth]{DTFTA1_Hz.png}
	\caption{DTFT do sinal {\tt A1.wav} plotado em função da frequência temporal (em Hz).}
	\label{fig:DTFTA1Hz}
\end{figure}

\subsection*{Tarefa 8}
Devido à periodicidade da DTFT, o espectro de amplitude se repete nos intervalos $[-3\pi, -\pi]$ e $[\pi,3\pi]$, conforme visto na Figura \ref{fig:PeriodicDTFT}.

\begin{figure}[h!]
	\centering
	\includegraphics[width = \textwidth,clip = true, trim = 0.5cm 0.2cm 0.5cm 0.2cm]{Periodic_DTFT.png}
	\vskip-14pt
	\caption{DTFT do sinal {\tt A1.wav} plotado no intervalo $[-3\pi,3\pi]$ rad/amostra.}
	\label{fig:PeriodicDTFT}
\end{figure}

Como o número de elementos em {\tt Omega} é o mesmo mas a faixa de frequência triplicou, a resolução em frequência é apenas um terço da resolução anterior, resultando em $\Delta f=\frac{3f_S}{4096 - 1}=16.15$ Hz $= 0.0046$ rad/amostra.

\section{DFT e IDFT}

\subsection*{Tarefa 1}

Mostre que a DFT é periódica

\begin{align*}
	X[k+N] & = \sum_{n=0}^{N-1} x[n] \cdot e^{-j\frac{2\pi}{N}(k+N)n}  \\
	& = \sum_{n=0}^{N-1} x[n] \cdot e^{-j\frac{2\pi}{N}kn} \cdot e^{-j 2\pi n} \\
	& = \sum_{n=0}^{N-1} x[n] \cdot e^{-j\frac{2\pi}{N}kn} \cdot \cancelto{1}{e^{-j 2\pi n}} \\
	& = \sum_{n=0}^{N-1} x[n] \cdot e^{-j\frac{2\pi}{N}kn} = X[k]
\end{align*}

Mostre que a DFT assume que $x[n]$ é periódica:

\begin{align*}
	x[n+N] & = \frac{1}{N}  \sum_{n=0}^{N-1} X[k] \cdot e^{j\frac{2\pi}{N}k(n+N)} \\
	& = \frac{1}{N}  \sum_{n=0}^{N-1} X[k] \cdot e^{j\frac{2\pi}{N}kn} \cdot e^{j2\pi k} \\
	& = \frac{1}{N}  \sum_{n=0}^{N-1} X[k] \cdot e^{j\frac{2\pi}{N}kn} \cdot \cancelto{1}{e^{j2\pi k}} \\
	& = \frac{1}{N}  \sum_{n=0}^{N-1} X[k] \cdot e^{j\frac{2\pi}{N}kn} = x[n]
\end{align*}

\subsection*{Tarefa 2}
-

\subsection*{Tarefa 3}
Vamos criar uma sequência de valores aleatória (com distribuição Gaussiana) e calcular a DFT para os três casos:

\begin{lstlisting}[frame=single]
# Gerar sequencia aleatoria
L = 20
ruido = np.random.randn(L)

# Usar DFTs de comprimento diferentes
DFT_Ruido1 = DFT(ruido, L)
DFT_Ruido2 = DFT(ruido, L-1)
DFT_Ruido3 = DFT(ruido, L+1)
\end{lstlisting}

\subsection*{Tarefa 4}
As três subfiguras na Figura \ref{fig:DFT} mostram o resultado para a magnitude e fase da DFT da sequência aleatória {\tt ruido} para os três valores de {\tt N\_DFT}:

\begin{figure}[h!]
	\centering
	\subfloat[$N_\textrm{DFT}=L$]{\label{subfig:DFT_L}
		%\includegraphics[width = 0.33\textwidth,clip = true,trim = 1cm 0cm 1cm 0cm]{DFT_L.png}
		\includegraphics[width = 0.33\textwidth]{DFT_L.png}
	}
	~
	\subfloat[$N_\textrm{DFT}=L-1$]{
		%\includegraphics[width = 0.33\textwidth,clip = true,trim = 1cm 0cm 1cm 0cm]{DFT_Lminus1.png}
		\includegraphics[width = 0.33\textwidth]{DFT_Lminus1.png}
	}
	~
	\subfloat[$N_\textrm{DFT}=L+1$]{
		%\includegraphics[width = 0.33\textwidth,clip = true,trim = 1cm 0cm 1cm 0cm]{DFT_Lplus1.png}
		\includegraphics[width = 0.33\textwidth]{DFT_Lplus1.png}
	}
	\caption{DFT da sequência aleatória {\tt ruido} para os três valores de  {\tt N\_DFT}.}
	\label{fig:DFT}
\end{figure}

Pode-se observar que a alteração do comprimento da DFT também altera o resultado da DFT, embora características claramente proeminentes possam permanecer muito semelhantes entre os dois resultados.

\subsection*{Tarefa 5}
%Como pode ser visto no gráfico de fase da Figura \ref{subfig:DFT_L}, o valor da fase para $\Omega = 0$ rad/amostra e $\Omega = \pi$ rad/amostra é zero. Pela definição da DFT, é evidente que, para sinais de valor real, a fase é $0$ ou $\pi$ para um $L$ par (Certifique-se de compreender o porquê!).
Pela definição, a DFT de uma sequência {\tt x} real, com L par, resultará em valores reais para $\Omega = 0$ rad/amostra e $\Omega = \pi$ rad/amostra. Portanto, a fase para estas amostras deve ser $0$ ou $\pm\pi$ radianos, como pode ser visto no gráfico da fase na Figura \ref{subfig:DFT_L}. Certifique-se de entender porquê isto acontece!

\subsection*{Tarefa 6}
A comparação entre as Figuras \ref{fig:DFT} e \ref{fig:FFT} mostra que o resultado da DFT para $N_\textrm{DFT}=L$ é o mesmo que o da FFT correspondente, como era esperado.

\begin{figure}[h!]
	\centering
	\includegraphics[width=0.85\textwidth]{FFT_L.png}
	\caption{Resultado da FFT da sequência aleatória para $N_\textrm{DFT}=L$.}
	\label{fig:FFT}
\end{figure}

\subsection*{Tarefa 7}
-

\subsection*{Tarefa 8}

Como pode ser visto na Figura \ref{fig:IDFT}, as sequências obtidas pela operação IDFT são todas iguais até $n=18$. O resultado da IDFT para o espectro calculado com $N_\textrm{DFT}=L-1$ termina na 19ª amostra, devido à limitação de dados previamente imposta. O resultado da IDFT para o comprimento de dados $L$ produz exatamente o mesmo resultado que a sequência original, e o resultado para o comprimento de dados $L=1$ adiciona um único zero à sequência original, visto que esse zero foi adicionado anteriormente durante a operação DFT com $N_\textrm{DFT}=L+1$.

\begin{figure}[h!]	
	\centering
	\includegraphics[width = 0.85\textwidth]{IDFT.png}
	\caption{Resultados da IDFT aplicada aos três espectros DFT diferentes.}
	\label{fig:IDFT}
\end{figure}

\section{DTFT vs. DFT}

\subsection*{Tarefa 1}

\begin{lstlisting}[frame = single]	
# Ler o sinal 'A1' entre 5100 e 6400 amostras
d = a1[5099:6400]
\end{lstlisting}

\subsection*{Tarefas 2-5}
Pode-se observar no trecho da Figura \ref{fig:DTFT_DFT} que os intervalos da DFT são ``amostras'' do resultado da DTFT, que é contínua em relação à frequência. A resolução de frequência da DFT depende claramente do número de amostras, enquanto a DTFT pode, em teoria, ter uma resolução de frequência infinita.

No entanto, também se observa que as 4096 amostras usadas para criar {\tt Omega} não fornecem resolução suficiente para que ambas as representações sejam visualmente alinhadas; um resultado muito melhor pode ser obtido recriando o vetor {\tt Omega} com cerca de 10000 pontos ou mais, embora isso não seja estritamente necessário.

\begin{figure}
	\centering
	\includegraphics[width = 0.85\textwidth]{DTFT_DFT.png}
	\caption{DTFT (azul) e DFT (hastes vermelhas) de {\tt d} para as frequências no intervalo $[-200,200]$ Hz.}
	\label{fig:DTFT_DFT}
\end{figure}

\end{document}
