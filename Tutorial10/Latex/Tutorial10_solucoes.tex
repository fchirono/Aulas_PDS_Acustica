\documentclass[a4paper,twoside,12pt]{article}
\usepackage[top=1.5cm,left=1.5cm,right=1.5cm,bottom=2cm]{geometry}

\usepackage[utf8]{inputenc}
\usepackage[T1]{fontenc}
\usepackage{lmodern}
\usepackage[brazilian]{babel}

\usepackage{graphicx}
\usepackage{subfig}
\usepackage{psfrag}
\usepackage{cite}
\usepackage[cmex10]{amsmath}
\usepackage{amssymb}
\usepackage{upgreek}
\usepackage{microtype}
\usepackage{titling}
\usepackage{courier}
\usepackage{url}
\usepackage{hyperref}
\hypersetup{
	colorlinks=true,
	linkcolor=blue,
	urlcolor=blue
}

% Use o pacote 'listings' para adicionar código e o pacote 'color' para gerar os destaques
\usepackage{listings}
\usepackage{color}
\definecolor{grey}{rgb}{0.6,0.6,0.6}
\definecolor{codegreen}{rgb}{0,0.6,0}
\definecolor{codegray}{rgb}{0.5,0.5,0.5}
\definecolor{codepurple}{rgb}{0.58,0,0.82}
\definecolor{backcolour}{rgb}{0.95,0.95,0.92}
\lstset{language=Python,				% definir para destaque de linguagem Python
	backgroundcolor=\color{backcolour},
	commentstyle=\color{codegreen},
	keywordstyle=\color{magenta},
	numberstyle=\tiny\color{codegray},
	stringstyle=\color{codepurple},
	basicstyle=\ttfamily\small,		% definir tamanho das fontes
	breaklines=true,				% definir quebra de linha automática
	linewidth=\textwidth,			% definir tamanho da caixa de código
	showstringspaces=false}			% mostrar espaços como sublinhados apenas dentro de strings


\title{\vspace{-2cm} Processamento Digital de Sinais\\ Soluções para Tutorial 10 - Espectro de Potência}
\author{\href{https://github.com/fchirono/AulasDSP}{Fabio Casagrande Hirono}}

\begin{document}
	\date{}
	\maketitle
	
	
	%\setcounter{section}{1}
	\section{Tarefas}
	
	\subsection{Tarefa 1}
	
	\begin{figure}[h!]
		\centering
		\includegraphics[width = 0.75\textwidth]{RespFreqT1.png}
		\caption{Magnitude da Resposta em Frequência conforme extraída do arquivo {\tt tutorial10.mat}.}
		\label{fig:RespFreqT1}
	\end{figure}
	
	\subsection{Tarefa 2}
	Complementar os dados da resposta em frequência fornecida de modo que corresponda ao resultado da DFT de um sinal no domínio do tempo de valor real produz a resposta ao impulso representada na Figura \ref{fig:RI_reconstruida}.

	\begin{figure}[h!]
		\centering
		\includegraphics[width = 0.75\textwidth]{RI_T2.png}
		\caption{As primeiras 11 amostras da resposta ao impulso conforme foi reconstruída a partir da resposta em frequência fornecida.}
		\label{fig:RI_reconstruida}
	\end{figure}

	Note que apenas as primeiras 11 amostras da resposta ao impulso estão representadas no gráfico de hastes.
	
	\subsection{Tarefa 3}
	-
	
	\subsection{Tarefa 4}
	-
	
	\subsection{Tarefa 5}
	A coerência entre o sinal de ruído branco {\tt x} e a saída do sistema {\tt y} é apresentada na Figura \ref{fig:Coerencia_T5}.
	
	\begin{figure}[h!]
		\centering
		\includegraphics[width = 0.75\textwidth]{Coerencia_T5.png}
		\caption{Função de coerência entre o ruído branco de entrada {\tt x} e o sinal na saída do sistema {\tt y}.}
		\label{fig:Coerencia_T5}
	\end{figure}
	
	A razão para a coerência despencar em torno de duas frequências além de $f=10$ kHz pode ser encontrada na resposta em frequência do sistema na Figura \ref{fig:RespFreqT1}. As duas frequências são precisamente aquelas onde o sistema tem a menor magnitude de resposta.
	
	\subsection{Tarefa 6}
	Pode-se observar nas Figuras \ref{fig:RespFreqT6} e \ref{fig:RespImpT6} que as estimativas mostram uma correspondência razoavelmente boa (embora não perfeita) com o sistema original.
	
	\begin{figure}[h!]
		\centering
		\subfloat[Respostas em frequência conforme fornecidas pelos dados originais e pela estimativa obtida pelo método H1.]{\label{fig:RespFreqT6}
			\includegraphics[width = 0.75\textwidth]{RespFreqT6.png}
		}
		\\
		\subfloat[Respostas ao impulso conforme fornecidas pelos dados originais e pela estimativa obtida pelo método H1.]{\label{fig:RespImpT6}
			\includegraphics[width = 0.75\textwidth]{RespImpT6.png}
		}
		\caption{Respostas ao Impulso e em Frequência fornecidas pelos dados originais e pelo estimador H1.}
	\end{figure}
	
	\subsection{Tarefa 7}
	Pode-se observar da estimativa obtida de dados ruidosos, que embora as primeiras 11 amostras da IR pareçam não ser muito diferentes da estimativa anterior feita a partir de dados limpos (veja Fig. \ref{fig:RespImpT6} e \ref{fig:RespImpT7}), a resposta em frequência (veja Fig. \ref{fig:RespFreqT7}) mostra impacto substancial. Isso pode ser explicado ao observar a IR identificada completa, que mostra muito ruído na parte da IR além da 11ª amostra. Precisamente este ruído é a razão para a qualidade degradada da resposta em frequência.
	
	\begin{figure}[h!]
		\centering
		\subfloat[Respostas em frequência conforme fornecidas pelos dados originais e estimadas com o método H1 com dados limpos e com dados ruidosos.]{
			\includegraphics[width = 0.75\textwidth]{RespFreqT7.png}
			\label{fig:RespFreqT7}
		}
		\\
		\subfloat[Estimativa de resposta ao impulso obtida pelo método H1 com dados ruidosos.]{
			\includegraphics[width = 0.75\textwidth]{RespImpT7.png}
			\label{fig:RespImpT7}
		}
		\caption{Respostas em Frequência e ao Impulso obtidas pelo estimador H1 com dados ruidosos.}
	\end{figure}
	
	\begin{figure}[h!]
		\centering
		\subfloat[Erro de Estimação da IR]{
			\includegraphics[width = 0.75\textwidth]{ErroEstimacaoT7.png}
		}
		\\
		\subfloat[Coerência entre entrada e saída ruidosa]{
			\includegraphics[width = 0.75\textwidth]{Coerencia_T7.png}
		}
		\caption{Erro de estimação da Resposta ao Impulso e Coerência obtidos pelo estimador H1 com dados ruidosos.}
	\end{figure}
	
	\subsection{Tarefa 8}
	Os resultados são exibidos nas Figuras \ref{fig:RespFreqT8} e \ref{fig:Coerencia_T8}. Como pode ser esperado, a estimação da resposta em frequência para {\tt M = 200} é razoavelmente boa. Para {\tt M=1000} o atraso relativamente grande da IR causa uma coerência ligeiramente pior entre os dados de entrada e saída, o que também afeta a precisão da estimativa da FRF. Para o caso de {\tt M=6000} a coerência despenca em direção a zero, mostrando que a estimativa da FRF obtida deve ser considerada nada mais que ruído. Isso é facilmente compreendido já que o atraso é tão grande que não será identificado dentro do quadro de observação dado. Isso também é o que causa a queda da coerência em direção a zero.
	\begin{figure}[h!]
		\centering
		\subfloat[Respostas em frequência obtidas da estimação H1 para vários valores de {\tt M}.]{\label{fig:RespFreqT8}
			\includegraphics[width = 0.75\textwidth]{RespFreqT8.png}
		}
		\\
		\subfloat[Coerência entre sinais de entrada e saída para vários valores de {\tt M}.]{\label{fig:Coerencia_T8}
			\includegraphics[width = 0.75\textwidth]{Coerencia_T8.png}
		}
		\caption{Efeito do parâmetro {\tt M}.}
	\end{figure}
	
\end{document}

