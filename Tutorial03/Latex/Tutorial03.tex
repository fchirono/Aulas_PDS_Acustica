\documentclass[a4paper,twoside,12pt]{article}
\usepackage[top=1.5cm,left=1.5cm,right=1.5cm,bottom=2cm]{geometry}

\usepackage[utf8]{inputenc}
\usepackage[T1]{fontenc}
\usepackage{lmodern}
\usepackage[brazilian]{babel}

\usepackage{graphicx}
\usepackage{subfig}
\usepackage{psfrag}
\usepackage{cite}
\usepackage[cmex10]{amsmath}
\usepackage{amssymb}
\usepackage{upgreek}
\usepackage{microtype}
\usepackage{titling}
\usepackage{courier}
\usepackage{url}
\usepackage{hyperref}
\hypersetup{
	colorlinks=true,
	linkcolor=blue,
	urlcolor=blue
}

% Use the 'listings' package to add code and the 'color' package to generate the highlights
\usepackage{listings}
\usepackage{color}
\definecolor{grey}{rgb}{0.6,0.6,0.6}
\definecolor{codegreen}{rgb}{0,0.6,0}
\definecolor{codegray}{rgb}{0.5,0.5,0.5}
\definecolor{codepurple}{rgb}{0.58,0,0.82}
\definecolor{backcolour}{rgb}{0.95,0.95,0.92}
\lstset{language=Python,				% set it to Python language highlighting
	backgroundcolor=\color{backcolour},
	commentstyle=\color{codegreen},
	keywordstyle=\color{magenta},
	numberstyle=\tiny\color{codegray},
	stringstyle=\color{codepurple},
	basicstyle=\ttfamily\small,		% set size of fonts
	breaklines=true,				% set automatic line breaking
	linewidth=\textwidth,			% set size of code box
	showstringspaces=false}			% show spaces as underscores only inside strings

\title{\vspace{-2cm} Processamento Digital de Sinais e Aplicações em Acústica\\ Tutorial 03 - Amostragem e Aliasing}
\author{\url{https://github.com/fchirono/Aulas_PDS_Acustica}}

\begin{document}
	\date{}
	\maketitle
	
	\section*{Objetivos do Tutorial}
	
		Ao final desta sessão, você será capaz de:
	
		\begin{itemize}
			\item Compreender a diferença entre sinais em tempo discreto e tempo contínuo
			\item Aplicar o Teorema de Shannon para reconstrução de sinais
			\item Identificar e visualizar o fenômeno de aliasing
			\item Reconhecer as limitações da frequência de amostragem
		\end{itemize}

	\begingroup
	\let\clearpage\relax
	\tableofcontents
	\endgroup
	
	
	\section{Conceitos Básicos}
	
	\subsection{Amostragem Temporal}
	
	A conversão de sinais analógicos (tempo contínuo) para digitais (tempo discreto) é realizada através da amostragem:
	
	\begin{equation}
		x[n] = x(t) | _{t = n \cdot T_s}, \ n \in [0, \ldots N-1],
	\end{equation}
	
	\noindent onde $T_s$ é o intervalo de amostragem em segundos e $f_s = 1/T_s$ é a frequência de amostragem em Hertz. O sinal discreto $x[n]$ representa o sinal contínuo $x(t)$ nos instantes $t = nT_s$.
	
	\subsection{Teorema de Nyquist-Shannon (Teorema da Amostragem)}
	
	Para que um sinal seja perfeitamente reconstruído a partir de suas amostras, a frequência de amostragem $f_s$ deve ser pelo menos duas vezes maior que a maior frequência presente no sinal:
	
	\begin{equation}
		f_s \geq 2f_{\text{max}}.
	\end{equation}
	
	Em outras palavras, o sinal deve ser de \emph{banda limitada}. A frequência $f_s/2$ é conhecida como \emph{frequência de Nyquist}, e define a largura de banda máxima para que um sinal seja amostrado corretamente. 
	
	Quando a condição do Teorema de Nyquist-Shannon não é satisfeita ($f_s < 2f_{\text{max}}$), componentes de alta frequência aparecem ``disfarçadas'' como componentes de baixa frequência no sinal amostrado. Este fenômeno é chamado de \textit{aliasing}.
	
	\subsection{Interpolação usando funções sinc}
	
	Assumindo um sinal de banda limitada amostrado sem \emph{aliasing}, a reconstrução ideal do sinal de tempo contínuo a partir de suas amostras discretas é dada por:
	
	\begin{equation}
		x(t) = \sum_{n=-\infty}^{\infty} x[n] \cdot \text{sinc}\left( \left( t - n \cdot T_s \right) \cdot f_s \right),
	\end{equation}
	
	\noindent onde $\text{sinc}(x) = \sin(\pi x)/(\pi x)$ é a função interpolante ideal. Esta função tem a propriedade importante de valer 1 em $x=0$ e zero em todos os outros valores inteiros de $x$.
	
	\section{Exercícios}
	
	\subsection{Visualização do sinal discreto vs contínuo}
	
	Crie um script que:
	
	\begin{enumerate}
		\item Defina uma frequência de amostragem $f_{s,d} = 1000$ Hz para tempo discreto
		\item Defina uma frequência muito maior $f_{s,c} = 20 \times f_{s,d}$ para simular tempo ``contínuo''
		\item Gere um sinal cosenoidal com frequência $f_0 = 53$ Hz e duração $T = 1.0$ s em ambas as frequências de amostragem
		\item Plote ambos os sinais (discreto com marcadores, contínuo com linha).
	\end{enumerate}
	
	\textbf{Questão:} O que você observa sobre a relação entre as amostras discretas e o sinal contínuo?
	% As amostras do sinal discreto ocorrem a intervalos igualmente espaçados sobre o sinal "continuo"
	
	\subsection{Propriedades da função sinc}
	
	%Demonstre numericamente que a interpolação tipo sinc de uma única amostra possui valor zero nos instantes das outras amostras:
	
	\begin{enumerate}
		\item Crie um sinal de tempo discreto com 1 segundo de duração, com valor unitário na 15a amostra e zero em todas as outras amostras
		\item Interpole esta amostra usando a função sinc para o tempo ``contínuo''
		\item Plote ambos os sinais, novamente com marcadores para o sinal discreto e com linhas para o sinal ``contínuo''
		\item Limite a visualização ao início do sinal para melhor visualização.
	\end{enumerate}
	
	\textbf{Questão:} Qual é a relação entre o sinal discreto e o sinal ``contínuo''? A função sinc é causal? O que isso implica para sistemas físicos reais?
	% A funcao tipo sinc(t-t0) interpola com unicamente a amostra no instante t=t0 com valor nao-zero, e com valor zero em todos os outros instantes do sinal discreto.
	% A funcao sinc NAO é causal, pois ela possui valores não-zero tanto antes quando depois do seu pico - ou seja, amostras no presente sofrem influencia de amostras no futuro. Desta forma, um sistema interpolante ideal tipo sinc não pode ser realizado em um sistema físico em tempo real.
	
	\subsection{Reconstrução ideal de sinais (Teorema de Shannon)}
	
	Implemente a reconstrução de um sinal usando o somatório de funções sinc:
	
	\begin{enumerate}
		\item Para as primeiras 15 amostras do sinal discreto do Exercício 1:
		\begin{itemize}
			\item Calcule a função sinc ``contínua'' correspondente a cada amostra
			\item Plote cada contribuição individual no tempo contínuo
			\item Some todas as contribuições para obter o sinal reconstruído
		\end{itemize}
		\item Compare visualmente o sinal reconstruído (``contínuo'') com o original (discreto)
	\end{enumerate}
	
%	\textbf{Questões:}
%	\begin{itemize}
%		\item Por que as funções sinc não interferem entre si nos instantes de amostragem?
%		\item Qual é a principal limitação prática do uso de funções sinc para interpolação?
%	\end{itemize}
	
	\subsection{Aliasing por sub-amostragem}
	
	Demonstre o \textit{aliasing} quando um sinal de alta taxa de amostragem é dizimado:
	
	\begin{enumerate}
		\item Gere sinais senoidais contínuos ($f_{s,c} = 20000$ Hz) com frequências 250, 450 e 750 Hz
		\item Para cada sinal, plote:
		\begin{itemize}
			\item O sinal em tempo ``contínuo''
			\item O sinal amostrado com amostras tomadas a cada 20 pontos (equivalente a $f_{s,d} = 1000$ Hz)
		\end{itemize}
		\item Para o sinal de 750 Hz, sobreponha um sinal senoidal de 250 Hz
		\item Use \texttt{sounddevice.play()} para auralizar o sinal em tempo ``contínuo'' em $f_{s,c}=20000$ Hz, e o sinal sub-amostrado em $f_{s,d} = 1000$.
	\end{enumerate}
	
		\textbf{Questões:}
	\begin{itemize}
		\item Sob quais condições o sinal ``contínuo'' e o sinal sub-amostrado resultam na mesma frequência?
	\end{itemize}
	
	
	\subsection{Visualização de Aliasing no Domínio do Tempo e Frequência}
	
	Para cada uma das frequências $f_1 = [150, 350, 550, 750]$ Hz:
	
	\begin{enumerate}
		\item Gere sinais cossenoidais amostrados a $f_{s,d} = 1000$ Hz
		\item Calcule a FFT de cada sinal
		\item Para cada sinal, crie uma figura com dois subplots:
		\begin{itemize}
			\item Superior: sinal discreto (marcadores) e ``contínuo'' (linha tracejada)
			\item Inferior: magnitude da FFT centrada em $f=0$ Hz, e réplicas espectrais em $\pm f_s$. Marque também as frequências $\pm f_1$ da onda senoidal com linhas verticais \\(Sugestão: use \tt{matplotlib.pyplot.vlines})
		\end{itemize}
		\item Para frequências acima de $f_s/2$, plote também o sinal senoidal da frequência ``alias'' correspondente no tempo ``contínuo''.
		
	\end{enumerate}
	
	\textbf{Questões:}
	\begin{itemize}
		\item Quais frequências sofrem \textit{aliasing}? Por quê?
		\item Qual é a relação entre a frequência real e sua versão ``aliased''?
	\end{itemize}
	
	
	\section{Observações Finais}
	
	\begin{itemize}
		\item Use \texttt{plt.grid()} para facilitar a leitura dos gráficos
		\item Sempre inclua legendas e rótulos nos eixos
		\item Experimente com diferentes valores de frequências e taxas de amostragem
	\end{itemize}
	
	
\end{document}
