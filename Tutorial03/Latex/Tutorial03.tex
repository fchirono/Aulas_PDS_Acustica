\documentclass[a4paper,twoside,12pt]{article}
\usepackage[top=1.5cm,left=1.5cm,right=1.5cm,bottom=2cm]{geometry}

\usepackage[utf8]{inputenc}
\usepackage[T1]{fontenc}
\usepackage{lmodern}
\usepackage[brazilian]{babel}

\usepackage{graphicx}
\usepackage{subfig}
\usepackage{psfrag}
\usepackage{cite}
\usepackage[cmex10]{amsmath}
\usepackage{amssymb}
\usepackage{upgreek}
\usepackage{microtype}
\usepackage{titling}
\usepackage{courier}
\usepackage{url}
\usepackage{hyperref}
\hypersetup{
	colorlinks=true,
	linkcolor=blue,
	urlcolor=blue
}

% Use the 'listings' package to add code and the 'color' package to generate the highlights
\usepackage{listings}
\usepackage{color}
\definecolor{grey}{rgb}{0.6,0.6,0.6}
\definecolor{codegreen}{rgb}{0,0.6,0}
\definecolor{codegray}{rgb}{0.5,0.5,0.5}
\definecolor{codepurple}{rgb}{0.58,0,0.82}
\definecolor{backcolour}{rgb}{0.95,0.95,0.92}
\lstset{language=Python,				% set it to Python language highlighting
	backgroundcolor=\color{backcolour},
	commentstyle=\color{codegreen},
	keywordstyle=\color{magenta},
	numberstyle=\tiny\color{codegray},
	stringstyle=\color{codepurple},
	basicstyle=\ttfamily\small,		% set size of fonts
	breaklines=true,				% set automatic line breaking
	linewidth=\textwidth,			% set size of code box
	showstringspaces=false}			% show spaces as underscores only inside strings

\title{\vspace{-2cm} Processamento Digital de Sinais e Aplicações em Acústica\\ Tutorial 03 - Amostragem e Aliasing}
\author{\url{https://github.com/fchirono/Aulas_PDS_Acustica}}

\begin{document}
	\date{}
	\maketitle
	
	\section*{Objetivos do Tutorial}
	
		Ao final desta sessão, você será capaz de:
	
		\begin{itemize}
			\item Compreender a diferença entre sinais em tempo discreto e tempo contínuo
			\item Aplicar o Teorema de Shannon para reconstrução de sinais
			\item Identificar e visualizar o fenômeno de aliasing
			\item Reconhecer as limitações da frequência de amostragem
		\end{itemize}

	\begingroup
	\let\clearpage\relax
	\tableofcontents
	\endgroup
	
	
	\section{Conceitos Básicos}
	
	\subsection{Amostragem Temporal}
	
	A conversão de sinais analógicos (tempo contínuo) para digitais (tempo discreto) é realizada através da amostragem a uma frequência $f_s = 1/T_s$, onde $T_s$ é o intervalo de amostragem. O sinal discreto $x[n]$ representa o sinal contínuo $x(t)$ nos instantes $t = nT_s$.
	
	\subsection{Teorema de Shannon (Teorema da Amostragem)}
	
	Para que um sinal seja perfeitamente reconstruído a partir de suas amostras, a frequência de amostragem $f_s$ deve ser pelo menos duas vezes maior que a maior frequência presente no sinal:
	
	\begin{equation}
		f_s \geq 2f_{\text{max}}
	\end{equation}
	
	A frequência $f_N = f_s/2$ é conhecida como frequência de Nyquist.
	
	\subsection{Interpolação usando funções sinc}
	
	A reconstrução ideal de um sinal contínuo a partir de suas amostras discretas é dada por:
	
	\begin{equation}
		x(t) = \sum_{n=-\infty}^{\infty} x[n] \cdot \text{sinc}((t - nT_s)f_s)
	\end{equation}
	
	onde $\text{sinc}(x) = \sin(\pi x)/(\pi x)$. Esta função tem a propriedade importante de valer 1 em $x=0$ e zero em todos os outros valores inteiros de $x$.
	
	\subsection{Aliasing}
	
	Quando a condição do Teorema de Shannon não é satisfeita ($f_s < 2f_{\text{max}}$), componentes de alta frequência aparecem ``disfarçadas'' como componentes de baixa frequência no sinal amostrado. Este fenômeno é chamado de \textit{aliasing}.
	
	\section{Exercícios}
	
	\subsection{Exercício 1: Visualização de Sinal Discreto vs Contínuo}
	
	Crie um script que:
	
	\begin{enumerate}
		\item Defina uma frequência de amostragem $f_{s,d} = 1000$ Hz para tempo discreto
		\item Defina uma frequência muito maior $f_{s,c} = 20 \times f_{s,d}$ para simular tempo ``contínuo''
		\item Gere um sinal cosenoidal com frequência $f_0 = 53$ Hz e duração $T = 1.0$ s
		\item Plote ambos os sinais (discreto com marcadores, contínuo com linha)
		\item Limite o eixo $x$ para visualizar apenas os primeiros 0.1 segundos
	\end{enumerate}
	
	\textbf{Questão:} O que você observa sobre a relação entre as amostras discretas e o sinal contínuo?
	
	\subsection{Exercício 2: Propriedade da Função sinc}
	
	Demonstre numericamente que a interpolação tipo sinc de uma única amostra possui valor zero nos instantes das outras amostras:
	
	\begin{enumerate}
		\item Crie um vetor contendo zeros exceto na 15ª posição
		\item Interpole esta amostra usando a função sinc no tempo ``contínuo''
		\item Plote o resultado mostrando que a função vale zero nas outras amostras
		\item Limite a visualização aos primeiros 0.03 segundos
	\end{enumerate}
	
	\textbf{Questão:} A função sinc é causal? O que isso implica para sistemas físicos reais?
	
	\subsection{Exercício 3: Reconstrução Ideal de Sinais (Teorema de Shannon)}
	
	Implemente a reconstrução de um sinal usando o somatório de funções sinc:
	
	\begin{enumerate}
		\item Para as primeiras 15 amostras do sinal discreto do Exercício 1:
		\begin{itemize}
			\item Calcule a função sinc correspondente a cada amostra
			\item Plote cada contribuição individual no tempo contínuo
			\item Some todas as contribuições para obter o sinal reconstruído
		\end{itemize}
		\item Compare visualmente o sinal reconstruído com o original
		\item Limite o eixo $x$ ao intervalo $[-0.005, 0.02]$ segundos
	\end{enumerate}
	
	\textbf{Questões:}
	
	\begin{itemize}
		\item Por que as funções sinc não interferem entre si nos instantes de amostragem?
		\item Qual é a principal limitação prática do uso de funções sinc para interpolação?
	\end{itemize}
	
	\subsection{Exercício 4: Visualização de Aliasing no Domínio do Tempo e Frequência}
	
	Para cada uma das frequências $f_1 = [150, 350, 550, 750]$ Hz:
	
	\begin{enumerate}
		\item Gere sinais cossenoidais amostrados a $f_{s,d} = 1000$ Hz
		\item Calcule a FFT de cada sinal
		\item Para cada sinal, crie uma figura com dois subplots:
		\begin{itemize}
			\item Superior: sinal discreto (marcadores) e contínuo (linha tracejada)
			\item Inferior: magnitude da FFT mostrando réplicas espectrais em $\pm f_s$
		\end{itemize}
		\item Para frequências acima de $f_s/2$, plote também a frequência ``alias''
	\end{enumerate}
	
	\textbf{Questões:}
	\begin{itemize}
		\item Quais frequências sofrem \textit{aliasing}? Por quê?
		\item Qual é a relação entre a frequência real e sua versão ``aliased''?
		\item Como as réplicas espectrais ajudam a visualizar o \textit{aliasing}?
	\end{itemize}
	
	\subsection{Exercício 5: Aliasing por Sub-amostragem}
	
	Demonstre o \textit{aliasing} quando um sinal de alta taxa de amostragem é dizimado:
	
	\begin{enumerate}
		\item Gere sinais contínuos ($f_{s,c} = 20000$ Hz) com frequências 250, 450 e 750 Hz
		\item Para cada sinal, plote:
		\begin{itemize}
			\item O sinal em tempo ``contínuo''
			\item Amostras tomadas a cada 20 pontos (equivalente a $f_{s,d} = 1000$ Hz)
		\end{itemize}
		\item Para o sinal de 750 Hz, sobreponha um cosseno de 250 Hz
	\end{enumerate}
	
	\textbf{Questão:} Por que os sinais de 250 Hz e 750 Hz produzem as mesmas amostras quando amostrados a 1000 Hz?
	
	\subsection{Exercício 6 (Opcional): Aliasing em Séries de Fourier}
	
	Crie um sinal composto por múltiplas componentes senoidais e observe o efeito do \textit{aliasing}:
	
	\begin{enumerate}
		\item Defina amplitudes $A = [1, 0.6, 0.4]$ e frequências $f = [150, 300, 600]$ Hz
		\item Crie um sinal somando as três componentes
		\item Compare os sinais em tempo discreto ($f_{s,d}$) e contínuo ($f_{s,c}$)
		\item Use \texttt{sounddevice.play()} para ouvir ambos os sinais (\textbf{OPCIONAL})
	\end{enumerate}
	
	\textbf{Questão:} Como o \textit{aliasing} afeta a percepção sonora do sinal?
	
	\section{Referências}
	
	K. Shin, J. Hammond. \textit{Fundamentals of Signal Processing for Sound and Vibration Engineers}. John Wiley and Sons, 2008.
	
	\section{Observações Finais}
	
	\begin{itemize}
		\item Use \texttt{plt.grid()} para facilitar a leitura dos gráficos
		\item Sempre inclua legendas e rótulos nos eixos
		\item Experimente com diferentes valores de frequências e taxas de amostragem
		\item Lembre-se: $f_s/2$ (frequência de Nyquist) é o limite teórico!
	\end{itemize}
	
	
\end{document}
