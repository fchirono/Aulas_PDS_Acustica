\documentclass[a4paper,twoside,12pt]{article}
\usepackage[top=1.5cm,left=1.5cm,right=1.5cm,bottom=2cm]{geometry}

\usepackage[utf8]{inputenc}
\usepackage[T1]{fontenc}
\usepackage{lmodern}
\usepackage[brazilian]{babel}
\usepackage{graphicx}
%\usepackage{subfig}
\usepackage{psfrag}
\usepackage{cite}
\usepackage[cmex10]{amsmath}
\usepackage{amssymb}
\usepackage{upgreek}
\usepackage{microtype}
\usepackage{titling}
\usepackage{courier}
\usepackage[caption=false]{subfig}  % pacote para subfloats

\usepackage{url}
\usepackage{hyperref}
\hypersetup{
	colorlinks=true,
	linkcolor=blue,
	urlcolor=blue
}

% Use the 'listings' package to add code and the 'color' package to generate the highlights
\usepackage{listings}
\usepackage{color}
\definecolor{grey}{rgb}{0.6,0.6,0.6}
\definecolor{codegreen}{rgb}{0,0.6,0}
\definecolor{codegray}{rgb}{0.5,0.5,0.5}
\definecolor{codepurple}{rgb}{0.58,0,0.82}
\definecolor{backcolour}{rgb}{0.95,0.95,0.92}
\lstset{language=Python,				% set it to Python language highlighting
	backgroundcolor=\color{backcolour},
	commentstyle=\color{codegreen},
	keywordstyle=\color{magenta},
	numberstyle=\tiny\color{codegray},
	stringstyle=\color{codepurple},
	basicstyle=\ttfamily\small,		% set size of fonts
	breaklines=true,				% set automatic line breaking
	linewidth=\textwidth,			% set size of code box
	showstringspaces=false}			% show spaces as underscores only inside strings


\title{\vspace{-2cm} Processamento Digital de Sinais e Aplicações em Acústica\\ Soluções para Tutorial 03 - Amostragem e Aliasing}
\author{\url{https://github.com/fchirono/Aulas_PDS_Acustica}}

\begin{document}
	\date{}
	\maketitle
	
	\setcounter{section}{2}
	\subsection{Visualização do sinal discreto vs contínuo}
	
	A Figura \ref{fig:Ex1_visualizacao} mostra o sinal discreto amostrado a $f_s,d=1000$ Hz com círculos vermelhos, e o sinal ``contínuo'' com uma linha cinza pontilhada. Nota-se que as amostras do sinal discreto são igualmente espaçadas no tempo, e coincidem com o sinal ``contínuo'' nos instantes amostrados.
	
	\begin{figure}[h]
		\centering
		\includegraphics[width=0.8\textwidth]{Ex1_visualizacao.png}
		\caption{Visualização de sinal senoidal de tempo ``contínuo'' e tempo discreto.}
		\label{fig:Ex1_visualizacao}
	\end{figure}
	
	\subsection{Propriedades da função sinc}
	
	A Figura \ref{fig:Ex2_sinc} mostra o sinal discreto com amostra unitária na 15a amostra ($t=15$ ms), e o sinal ``contínuo'' interpolado desta amostra através da função sinc. Nota-se que a função $\mathrm{sinc}(x-x_0)$ assume valor zero em todas as amostras discretas exceto a amostra $x=x_0$, sendo assim uma opção natural para interpolação de sinais discretos.
	
	A função sinc \textbf{NÃO} é causal, pois valores no presente são afetados por valores no futuro - note como o sinal ``contínuo'' apresenta oscilações mesmo antes do pico aos 15 ms. Como sistemas físicos reais devem ser causais, não é possível realizar uma interpolação ou reconstrução digital-analógica utilizando funções sinc ideais. Porém, a função sinc pode ser utilizada em pós-processamento de dados
	
	\begin{figure}[h]
		\centering
		\includegraphics[width=0.8\textwidth]{Ex2_sinc.png}
		\caption{Visualização da função sinc em tempo ``contínuo'' e tempo discreto.}
		\label{fig:Ex2_sinc}
	\end{figure}
		
	\subsection{Reconstrução ideal de sinais (Teorema de Shannon)}
	
	A Figura \ref{fig:Ex3_reconstrucao} mostra a reconstrução do sinal em tempo contínuo através das amostras discretas utilizando funções sinc. Nota-se que a soma das funções sinc se aproxima do sinal senoidal original, interpolando o valor da função de tempo contínuo nos períodos entre as amostras discretas.
	
	\begin{figure}[h]
		\centering
		\includegraphics[width=0.8\textwidth]{Ex3_reconstrucao.png}
		\caption{Visualização da reconstrução de sinal ``contínuo'' a partir das amostras em tempo discreto.}
		\label{fig:Ex3_reconstrucao}
	\end{figure}

	\subsection{Aliasing por sub-amostragem}
	
	A Figura \ref{fig:Ex4_subamostragem} demonstra os sinais amostrados a $f_{s,c}=20000$ Hz e $f_{s,d=1000}$ Hz. O sinal de 750 Hz possui frequência acima de $f_{s,d}/2$, e portanto sua versão discreta não obedece o Teorema da Amostragem e apresenta \emph{aliasing} - note como este sinal e o sinal de 250 Hz possuem a mesma representação em tempo discreto.
	
	Ao auralizar os sinais usando {\tt sounddevice.play} (lembre-se de usar as frequências de amostragem corretas!), não será possível notar diferença alguma nos primeiros dois casos, já que ambos obedecem o Teorema da Amostragem no caso discreto e portanto não há perda de informação. Já o terceiro sinal, que apresenta aliasing, não pode ser reproduzido corretamente, e ouvimos então o resultado do aliasing: perde-se a frequência correta do sinal amostrado, e apresenta-se uma outra frequência mais baixa.
	
	\begin{figure}[h]
		\centering
		\includegraphics[width=0.8\textwidth]{Ex4_subamostragem.png}
		\caption{Visualização de aliasing.}
		\label{fig:Ex4_subamostragem}
	\end{figure}
	
	\subsection{Visualização de Aliasing no Domínio do Tempo e Frequência}
	
	A Figura \ref{fig:Ex5_aliasing} demonstra o mesmo fenômeno de aliasing, mas desta vez representando os domínios do tempo e da frequência simultaneamente. O espectro em frequência azul demonstra frequências centradas em $f=0$, enquanto as réplicas espectrais em $\pm f_s$ são indicadas em linhas cinza tracejadas. Finalmente, a frequência $f_1$ do sinal senoidal está indicada com linhas verticais pretas.
	
	Nota-se aqui que quando a frequência $f_1$ encontra-se dentro do domínio $[-f_s/2, +f_s/2]$, o sinal amostrado possui a mesma frequência do sinal contínuo. Quando a frequência $f_1$ sai deste domínio e entra no domínio das réplicas espectrais, o sinal passa a apresentar \emph{aliasing} devido às componentes de frequência originalmente pertencentes às réplicas espectrais que adentram o domínio original centrado em $f=0$. Nota-se, assim, que a frequência real $f_1$ e a frequência da componente ``aliased'' $f_a$ estão relacionadas da forma $f_a = |f_1 - m \cdot f_s|$, onde $m$ é um número inteiro.
	
	\begin{figure}[h]
		\centering
		\subfloat[]{
			\includegraphics[width=0.5\textwidth]{Ex5_aliasing_150Hz.png}
			\label{fig:Ex5_aliasing_150Hz}
			}
		\subfloat[]{
			\includegraphics[width=0.5\textwidth]{Ex5_aliasing_350Hz.png}
			\label{fig:Ex5_aliasing_350Hz}
			} \\
		\subfloat[]{
			\includegraphics[width=0.5\textwidth]{Ex5_aliasing_550Hz.png}
			\label{fig:Ex5_aliasing_550Hz}
			}
		\subfloat[]{
			\includegraphics[width=0.5\textwidth]{Ex5_aliasing_750Hz.png}
			\label{fig:Ex5_aliasing_750Hz}
			}
		\caption{Sinais senoidais no tempo e na frequência: \protect\subref{fig:Ex5_aliasing_150Hz} $f_1 = 150$ Hz; \protect\subref{fig:Ex5_aliasing_350Hz}  $f_1 = 350$ Hz; \protect\subref{fig:Ex5_aliasing_550Hz}  $f_1 = 550$ Hz; e \protect\subref{fig:Ex5_aliasing_750Hz}  $f_1 = 750$ Hz;.}
		\label{fig:Ex5_aliasing}
	\end{figure}
	
\end{document}
