\documentclass[a4paper,twoside,12pt]{article}
\usepackage[top=1.5cm,left=1.5cm,right=1.5cm,bottom=2cm]{geometry}

\usepackage[utf8]{inputenc}
\usepackage[T1]{fontenc}
\usepackage{lmodern}
\usepackage[brazilian]{babel}
\usepackage{graphicx}
%\usepackage{subfig}
\usepackage{psfrag}
\usepackage{cite}
\usepackage[cmex10]{amsmath}
\usepackage{amssymb}
\usepackage{upgreek}
\usepackage{microtype}
\usepackage{titling}
\usepackage{courier}
\usepackage[caption=false]{subfig}  % pacote para subfloats

\usepackage{url}
\usepackage{hyperref}
\hypersetup{
	colorlinks=true,
	linkcolor=blue,
	urlcolor=blue
}

% Use the 'listings' package to add code and the 'color' package to generate the highlights
\usepackage{listings}
\usepackage{color}
\definecolor{grey}{rgb}{0.6,0.6,0.6}
\definecolor{codegreen}{rgb}{0,0.6,0}
\definecolor{codegray}{rgb}{0.5,0.5,0.5}
\definecolor{codepurple}{rgb}{0.58,0,0.82}
\definecolor{backcolour}{rgb}{0.95,0.95,0.92}
\lstset{language=Python,				% set it to Python language highlighting
	backgroundcolor=\color{backcolour},
	commentstyle=\color{codegreen},
	keywordstyle=\color{magenta},
	numberstyle=\tiny\color{codegray},
	stringstyle=\color{codepurple},
	basicstyle=\ttfamily\small,		% set size of fonts
	breaklines=true,				% set automatic line breaking
	linewidth=\textwidth,			% set size of code box
	showstringspaces=false}			% show spaces as underscores only inside strings


\title{\vspace{-2cm} Processamento Digital de Sinais e Aplicações em Acústica\\ Soluções para Tutorial 03 - Amostragem e Aliasing}
\author{\url{https://github.com/fchirono/Aulas_PDS_Acustica}}

\begin{document}
	\date{}
	\maketitle
	
	\setcounter{section}{2}
	\subsection{Visualização do sinal discreto vs contínuo}
	
	\begin{figure}
		\centering
		\includegraphics[width=0.8\textwidth]{Ex1_visualizacao.png}
		\caption{Visualização de sinal senoidal de tempo ``contínuo'' e tempo discreto.}
		\label{fig:Ex1_visualizacao}
	\end{figure}
	
	\subsection{Propriedades da função sinc}
	
	\begin{figure}
		\centering
		\includegraphics[width=0.8\textwidth]{Ex2_sinc.png}
		\caption{Visualização da função sinc em tempo ``contínuo'' e tempo discreto.}
		\label{fig:Ex2_sinc}
	\end{figure}
		
	\subsection{Reconstrução ideal de sinais (Teorema de Shannon)}
	
	\begin{figure}
		\centering
		\includegraphics[width=0.8\textwidth]{Ex3_reconstrucao.png}
		\caption{Visualização da reconstrução de sinal ``contínuo'' a partir das amostras em tempo discreto.}
		\label{fig:Ex3_reconstrucao}
	\end{figure}

	\subsection{Aliasing por sub-amostragem}
	
	\begin{figure}
		\centering
		\includegraphics[width=0.8\textwidth]{Ex4_subamostragem.png}
		\caption{Visualização de aliasing.}
		\label{fig:Ex4_subamostragem}
	\end{figure}
	
	\subsection{Visualização de Aliasing no Domínio do Tempo e Frequência}
	
	
	\begin{figure}
		\centering
		\subfloat[]{
			\includegraphics[width=0.5\textwidth]{Ex5_aliasing_150Hz.png}
			\label{fig:Ex5_aliasing_150Hz}
			}
		\subfloat[]{
			\includegraphics[width=0.5\textwidth]{Ex5_aliasing_350Hz.png}
			\label{fig:Ex5_aliasing_350Hz}
			} \\
		\subfloat[]{
			\includegraphics[width=0.5\textwidth]{Ex5_aliasing_550Hz.png}
			\label{fig:Ex5_aliasing_550Hz}
			}
		\subfloat[]{
			\includegraphics[width=0.5\textwidth]{Ex5_aliasing_750Hz.png}
			\label{fig:Ex5_aliasing_750Hz}
			}
		\caption{Sinais senoidais no tempo e na frequência: \protect\subref{fig:Ex5_aliasing_150Hz} $f_1 = 150$ Hz; \protect\subref{fig:Ex5_aliasing_350Hz}  $f_1 = 350$ Hz; \protect\subref{fig:Ex5_aliasing_550Hz}  $f_1 = 550$ Hz; e \protect\subref{fig:Ex5_aliasing_750Hz}  $f_1 = 750$ Hz;.}
		\label{fig:Ex5_aliasing}
	\end{figure}
	
\end{document}
