\section{Sistemas}

\subsection{Sistemas Variantes/Invariantes no Tempo}
O objetivo deste exercício é verificar se o sistema dado na função {\tt sistema1} (fornecida no módulo {\tt tutorial1\_funcoes.py} no Github) é invariante no tempo ou não. A sintaxe para a função é:

\begin{lstlisting}[frame=single]
y = Tutorial1.sistema1(x)
\end{lstlisting}

onde {\tt x} é o sinal de entrada do sistema, e {\tt y} é o sinal de saída do sistema.

\paragraph{Tarefas}

\begin{enumerate}
	\item Crie um sinal de impulso de amplitude unitária {\tt x\_1} com comprimento total de {\tt N = 100} amostras, e uma segunda versão atrasada no tempo {\tt x\_2} com o mesmo comprimento da primeira, mas atrasada em {\tt N\_0 = 50} amostras;
	\item Use os impulsos unitários e analise seus respectivos sinais de saída para determinar se o sistema {\tt Tutorial1.system1(x)} é invariante ou variante no tempo.
\end{enumerate}

\subsection{Sistemas Não Lineares}
\subsubsection{Revisão}
Um sistema $H$ é dito linear se ele satisfaz o princípio da superposição  -- ou seja, satisfaz aditividade e homogeneidade:

\begin{equation*}
	H(a_1 x_1[n] + a_2 x_2[n]) = a_1 H(x_1[n]) + a_2 H(x_2[n]).
\end{equation*}

Diagrama de blocos:\\

\begin{center}
	\includegraphics[width = 0.6\textwidth]{SistemaLinear.png}
\end{center}

\paragraph{Tarefas}
\begin{enumerate}
	\item Use um conjunto de sinais para descobrir se o sistema {\tt Tutorial1.sistema(x)} é linear ou não.
\end{enumerate}
